\documentclass[12pt,margin=0px]{article}

\usepackage[a4paper, margin=1in]{geometry}
\usepackage[hungarian]{babel}
\usepackage[normalem]{ulem}
\usepackage[utf8]{inputenc}
\usepackage{amsmath}
\usepackage{amssymb}
\usepackage{fontawesome}
\usepackage{enumitem}
\usepackage{fancyhdr}
\usepackage{listings}
\usepackage{makecell}
\usepackage{tikz}
\usetikzlibrary{positioning,calc,shapes.multipart,arrows,arrows.meta,matrix,automata,shapes.misc}

\newcommand\ddfrac[2]{\frac{\displaystyle #1}{\displaystyle #2}}
\DeclareMathOperator\arcsinh{arcsinh}
\DeclareMathOperator\arccosh{arccosh}
\DeclareMathOperator\sh{sinh}
\DeclareMathOperator\ch{cosh}

\geometry{
	a4paper,
	total={170mm,257mm},
	left=20mm,
	right=20mm,
	top=20mm,
	bottom=20mm
}

\setlist[itemize,1]{label=$\bullet$}
\setlist[itemize,2]{label=$\circ$}
\setlist[itemize,3]{label=$\centerdot$}
\setlist[itemize,4]{label=$\cdot$}

\pagestyle{fancy}

\newcommand\blfootnote[1]{%
	\begingroup
	\renewcommand\thefootnote{}\footnote{#1}%
	\addtocounter{footnote}{-1}%
	\endgroup
}

\renewcommand{\figurename}{ábra}

\newcommand{\N}{\mathbb{N}}
\newcommand{\Z}{\mathbb{Z}}
\newcommand{\R}{\mathbb{R}}
\newcommand{\Q}{\mathbb{Q}}
\newcommand{\C}{\mathbb{C}}

\makeatletter
\renewcommand\paragraph{%
	\@startsection{paragraph}{4}{0mm}%
	{-\baselineskip}%
	{.5\baselineskip}%
	{\normalfont\normalsize\bfseries}}
\makeatother

\useunder{\uline}{\ul}{}
\fancyhead{}
\cfoot{11. tétel | \thepage. oldal}

\renewcommand{\headrulewidth}{0pt}
\renewcommand{\footrulewidth}{0.4pt}

\tikzset{declare function={y(\x)=\x^2;},
	plot fill/.style={fill=white!75},
	plot/.style={draw=black!80, thick},
	bar/.style={fill=cyan, draw=white, thick},
	marking/.style={fill=cyan!50!black, draw=cyan!50!black},
	axis/.style={thick, draw=black!65, stealth-stealth}
}

\begin{document}
    \thispagestyle{fancy}
    \hyphenation{oddword}
    \uchyph=0
    {\Large\bfseries\noindent 11. Formális nyelvek} \\
	
    \section*{Formális nyelvtanok és a Chomsky-féle nyelvosztályok}
	
    {\footnotesize \noindent {\color{blue} \faLightbulbO\ $\triangleright$ } }

    \subsection*{Alapfogalmak\\}
			
    \subsubsection*{Ábécé, szó\\}
	
    \begin{itemize}
    	\item \textbf{\emph{Ábécének}} nevezzük egy tetszőleges véges szimbólumhalmazt.
        \item Az ábécé elemeit \textbf{\emph{betűk}}nek hívjuk.
        \item Egy $V$ ábécé elemeiből képzett \emph{véges sorozatokat} $V$ feletti \textbf{\emph{szavaknak}}.
        \item Adott $v \in V$ \textbf{\emph{szó hossza}} a benne szereplő betűk száma. Jelölése: $l(v)$ vagy $|v|$.
        \item A $V$ ábécé feletti $\varepsilon$ szót \textbf{\emph{üres szó}}nak nevezzük, ha $l(\varepsilon) = 0$.
        \item $V^{0} := \{\varepsilon \}$
        \item $V^{n} := V \times V^{n - 1},\ n \geq 1$
        \item A $V$ ábécé feletti szavak halmazát (beleértve az üres szót is) $\boldsymbol{V^*}$-gal jelöljük.\\
            $V^{*} := \bigcup\limits_{i = 0}^{\infty}V^{i} = V^{0} \cup V^{1} \cup \ldots$ a V \textbf{\emph{ábécé lezártja}}.
        \item A nemüres szavak halmazát $\boldsymbol{V^+}$-szal jelöljük ($V^{+} := V^{*}\backslash\{\varepsilon\}$).\\
            $V^{+} := \bigcup\limits_{i = 1}^{\infty}V^{i} = V^{1} \cup V^{2} \cup \ldots$ a V \textbf{\emph{ábécé pozitív lezártja}}.\\

        Példa: $V:= \{a, b\}$\\
        $V^{*} := \{\varepsilon, a, b, aa, ab, ba, bb, aaa, aab, \ldots\}$\\
        $V^{+} := \{a, b, aa, ab, ba, bb, aaa, aab, \ldots\}$
    \end{itemize}
						
    \subsubsection*{Szavak konkatenációja\\}

    \noindent Legyenek \emph{u} és \emph{v} szavak egy V ábécé felett. Ekkor a két szó \textbf{\emph{konkatenációjának}} nevezzük azt a szimbólumsorozatot, amelyet a két szó szimbólumainak egymás után fűzésével kapunk. Jelölése: \emph{uv}.\\

    \noindent \emph{Példa}: $V = \{a,b,c\},\ u = abb, v = cbb$. Ekkor $uv = abbcbb$.\\
						
    \noindent A konkatenáció asszociatív, de nem kommutatív művelet, melynek egységeleme $\varepsilon$.\\

    \noindent Így $V^{*}$ a konkatenációval mint művelettel és $\varepsilon$-nal egységelemes félcsoportot alkot.
    \begin{itemize}						
    	\item \noindent $V^*$ zárt a konkatenációra.
    	\begin{align*}
            u,v \in V^* \Rightarrow uv \in V^*
    	\end{align*}

    	\item $\varepsilon$ egységelem $V^*$-gal és a konkatenációval:

    	\begin{align*}
    		u \in V^* \Rightarrow u\varepsilon \in V^* \textrm{, és }\varepsilon u \in V^*
    	\end{align*}
    \end{itemize}

    \noindent Legyen $v \in V^{*}$
    \begin{itemize}
      \item $v^{0} = \varepsilon$ (bármely szó nulladik hatványa az üres szó)
      \item $v^{n} = vv^{n-1}\quad (n \geq 1)$ (bármely szó n. hatványa a szó n-szeres konkatenációja)
    \end{itemize}

	\subsubsection*{Egyéb definíciók}
	
	\begin{itemize}
		\item $u,v \in V$. Az $u$ szót a $v$ \textbf{\emph{részszavának}} nevezzük, ha $ \textbf{v} = x\textbf{u}y , \ (x,y \in V)$ teljesül.\\
        Ha még $xy\neq\varepsilon$, akkor $u$ \textbf{\emph{valódi részszó}}.

        \emph{Példa}: $v = aabbbcc \in V$ Az $u = abbbc$ valódi részszava $v$-nek.

		\item $ v  = xuy \ (x,y,u,v \in V)$. Ekkor:
		      \begin{itemize}
		      	\item Ha $x=\varepsilon$, akkor $u$-t a $v$ szó \textbf{\emph{prefix}}ének nevezzük\\
                \emph{Példa}: $v = aabbbcc$. Az $u = aabbb$ szó prefixe $v$-nek.
		      	\item Ha $y=\varepsilon$, akkor $u$-t a $v$ szó \textbf{\emph{szufix}}ének nevezzük\\
                \emph{Példa}: $v = aabbbcc$. Az $u = bbbcc$ szó szufixe $v$-nek.
		      \end{itemize}
		      						
		\item Egy $u\in V$ szó tükörképe alatt a szimbólumai fordított sorrendben való felírását értjük. (Jelölése: $u^{-1}$)
        \noindent Legyen $u = a_1 \ldots a_n,\ a_i \in V,\ 1 \leq i \leq n$, ekkor
        \[
            u^{-1} = a_n \ldots a_1
        \]
	\end{itemize}

    \noindent {\footnotesize $\triangleleft$ \faLightbulbO }
					
	\subsection*{Formális nyelvtanok és a Chomsky-féle nyelvosztályok}
					
	\paragraph*{Nyelv}
	
	\noindent A $V^{*}$ valamely részhalmazát ($2^{V^{*}}$ valamely elemét) a $V$ ábécé feletti \textbf{\emph{nyelv}}nek nevezzük. Jelölése: L. $(L \subset V^*)$.\\
							
	\noindent Az üres nyelv (egy szót sem tartalmaz) jele: $\emptyset$\\

	\noindent $L$ nyelv \textbf{\emph{véges nyelv}}, ha véges számú szót tartalmaz, különben \textbf{\emph{végtelen nyelv}}.\\

    \noindent $L_1 = \Big\{a,b,\varepsilon \Big\}$ \emph{véges} nyelv, $L_2 = \Big\{a^ib^i\ \Big|\ 0 \leq i\Big\}$ \emph{végtelen} nyelv.\\

    \noindent \emph{Megjegyzés}: A  formális  nyelv  nem  más,  mint  egy  adott  ABC  jeleiből  alkotott  tetszőleges  hosszú  szavak  halmazának  részhalmaza,  vagyis  a  formális  nyelv  egy adott ABC jeleiből alkotható, meghatározott szavak halmaza. A  formális  nyelv  állhat  véges  sok  szóból,  állhat  végtelen  sok  szóból  és  tartalmazhatja az üres szót is.\\

    \noindent Nyelvek valamely összességét nyelvosztálynak, nyelvcsaládnak hívjuk. Jelölése: $\mathcal{L}_i$ .\\
							
	\paragraph*{Nyelvekre vonatkozó műveletek}

	\begin{itemize}
		\item $ L_1 \cup L_2 = \Big\{\ u\ |\ \ u \in L_1 \vee u \in L_2\ \Big\}$ : az $L_1$ és $L_2$ nyelv uniója
		\item $ L_1 \cap L_2 = \Big\{\ u\ |\ \ u \in L_1 \wedge u \in L_2\ \Big\}$ : az $L_1$ és $L_2$ nyelv metszete
		\item $ L_1 - L_2 = \Big\{\ u\ |\ \ u \in L_1 \wedge u \notin L_2\ \Big\}$ : az $L_1$ és $L_2$ nyelv különbsége
		\item $\overline{L} = V^* - L$ : az $L \subseteq V^*$ komplementere
		\item $L_1L_2 = \Big\{\ u_1u_2\ |\ \ u_1 \in L_1, u_2 \in L_2\ \Big\}$ : az $L_1$ és $L_2$ nyelv konkatenációja\\

        Minden nyelvre fennáll: $\emptyset L = L \emptyset = \emptyset$ illetve $\{\varepsilon\}L = L\{\varepsilon\} = L$
		\item $L^i$ : az $L$ nyelv $i$-edik hatványa, és
        \begin{itemize}
            \item $L^0 = \{\varepsilon\}$
            \item $L^i = L^{i-1}L \qquad (i \geq 1)$
        \end{itemize}
        \item $ L^* = \bigcup\limits_{i \geq 0} L^i$ : az $L$ nyelv iteratív lezártja
        \item $ L^+ = \bigcup\limits_{i \geq 1} L^i$ nyelvet értjük
	\end{itemize}

	\noindent Az \emph{unió}, \emph{konkatenáció} és \emph{iteráció lezárása} műveleteket \emph{reguláris műveletek}nek nevezzük.

    \paragraph*{Nyelvekre vonatkozó leképezések}

    \noindent Legyen $V_1,\ V_2$ ábécé.

    \noindent A $h: V_1 \to V_2$ leképezést \textbf{\emph{homomorfizmus}}nak nevezzük, ha
    \[
        h(uv) = h(u)h(v) \qquad \forall u,v \in V_1^{*}
    \]

    \noindent Valamint minden $u = a_1a_2\ldots a_n$ szóra, ahol $a_i \in V,\ 1 \leq i \leq n$ fenáll, hogy
    \[
        h(u) = h(a_1)h(a_2)\ldots h(a_n)
    \]

    \noindent Legyen $h: V_1 \to V_2$ homomorfizmus.
    \noindent A $h$ \textbf{\emph{homomorfizmus $\varepsilon$-mentes}}, ha
    \[
        h(u) \neq \varepsilon,\ \forall u \in V_1^{*}\ \text{szóra},\ \text{ahol}\ u \neq \varepsilon
    \]

    \noindent Egy $h$ homomorfizmust \textbf{\emph{izomorfizmus}}nak nevezzük, ha bármely $\forall u, v \in V_1^{*}$  szóra teljesül, hogy
    \[
        h(u) = h(v) \Rightarrow u = v
    \]

    \noindent \emph{Példa}: izomorfizmus (decimális számok bináris reprezentációja):
    \begin{center}
        $V_1 = \Big\{0,1,2,\ldots,9\Big\},\ V_2 = \Big\{0,1\Big\}$\\
        $h(0) = 0000,\ h(1) = 0001,\ \ldots,\ h(9) = 1001$
    \end{center}

    \paragraph*{Nyelvek megadása}

    \noindent A formális nyelvek megadási módszereitől elvárjuk, hogy a leírás véges legyen.\\

    \noindent A formális nyelv megadható:
    \begin{enumerate}
        \item felsorolással (csak véges nyelvek esetén).
        \item a szavakat alkotó szabály szöveges leírásával.\\
        Például: \emph{Legyen $L_2$ a páratlan számok nyelve}.
        \item a szavakat alkotó szabály matematikai leírásával.\\
        Például: $V := \{a, b\}$ és legyen $L := \{u\ |\ u = a^{n}b^{n} \wedge n \in \mathbb{N}\}$
        \item generatív grammatika segítségével.
    \end{enumerate}

	\subsection*{Grammatika}

    \noindent A nyelveket megadhatjuk nyelvtanukkal is, vagyis egy szabályrendszerrel, aminek a felhasználásával a nyelv mondatai levezethetők. Egy generatív nyelvtan a jelsorozatok transzformációs szabályait leíró szabályok halmazából áll. A nyelvet alkotó jelsorozatok létrehozásához szükséges, hogy legyen egy egyedi „kezdő” szimbólum, ezután csak a szabályokat kell egymás után alkalmazni (bárhányszor, tetszés szerinti sorrendben) a kezdő szimbólum átalakítására. A nyelv azokból a jelsorozatokból áll, amelyeket az említett módon elő lehet állítani.\\

    \noindent Tegyük fel például, hogy egy ábécéhez a 'a' és a 'b' szimbólumok tartoznak, a kezdő szimbólum pedig legyen az 'S' és adottak a következő szabályok:
    \begin{itemize}
      \item $S \to aSb$
      \item $S \to ba$
    \end{itemize}	

    \noindent Kezdő szimbólumunk az 'S', de ezután kiválaszthatjuk, hogy melyik szabályt alkalmazzuk a jelsorozat következő elemének előállításához. \\

    \noindent \emph{Példa}:
    \begin{itemize}
        \item Ha az 1-es szabályt választjuk, akkor annak alapján az 'S' szimbólumot a 'aSb'-al helyettesítjük, eredményül tehát a "aSb" jelsorozatot kapjuk. \\
        \item Ha most ismételten az 1-es szabály alkalmazását választjuk, akkor helyettesítjük az 'S' szimbólumot a 'aSb'-vel, és akkor a "aaSbb" jelsorozatot kapjuk.\\
        \item Ezt az eljárást addig ismételhetjük, amíg az ábécé szimbólumai megengedik.\\
        \item Befejezve a példát, most válasszuk a 2-es szabályt, helyettesítsük az 'S' szimbólumot a 'ba' jelsorozattal, eredményül pedig a "aababb" jelsorozatot kapjuk, és ezzel be is fejeztük.
    \end{itemize}

    \noindent A nyelvtan által meghatározott nyelv nem lesz más, mint az összes olyan jelsorozat halmaza, amelyeket ezzel az eljárással elő tudunk állítani:
    \[
        \Big\{ba, abab, aababb, aaababbb, \ldots \Big\}
    \]
\newpage
	\noindent A $G$ generatív nyelvtan egy $\Big<N,T,\mathcal{P},S\Big>$ négyest értünk, ahol:
	\begin{itemize}
		\item $N$ és $T$ diszjunkt ábécék, a nemterminális ($N$) és terminális ($T$) szimbólumok ábécéi.
		\item $S\in N$ a kezdőszimbólum.
		\item $\mathcal{P}$ az $(x,y)$ rendezett párok halmaza, ahol $x,y \in (N \cup T)^*$ és $x$ legalább egy nemterminális szimbólumot tartalmaz. \\
		      A $\mathcal{P}$ halmaz elemeit átírási szabályoknak nevezzük.\\
	\end{itemize}

    \noindent \emph{Megjegyzés}: $N\ \cap\ T = \emptyset$ és az $(x,y)$ jelölés helyett az $x \rightarrow y$ jelölést alkalmazzuk.\\

    \noindent \textbf{Példa}: $N = \Big\{S, B\Big\},\ T = \Big\{a, b, c\Big\}$\\
    $\mathcal{P}$:
    \begin{itemize}
        \item[(1)] $S \to aBSc$
        \item[(2)] $S \to abc$
        \item[(3)] $Ba \to aB$
        \item[(4)] $Bb \to bb$
    \end{itemize}
    És $S \in N$ nemterminális kezdőszimbólum.\\

    \noindent Ekkor $L(G)$, a G grammatika által generált nyelvben generált néhány szimbólumsorozat.
    \begin{itemize}
        \small
        \item[] $S \to (2)\ abc$
        \item[] $S \to (1)\ aB\textbf{S}c \to (2)\ a\textbf{Ba}bcc \to\ (3)\ aa\textbf{Bb}cc \to (4)\ aabbcc$
        \item[] $S \to (1)\ aB\textbf{S}c \to (1)\ aBaB\textbf{S}cc \to (2)\ a\textbf{Ba}Babccc \to\ (3)\ aaB\textbf{Ba}bccc \to (3)\ aa\textbf{Ba}Bbccc \to (3)\ aaaB\textbf{Bb}ccc \to (4)\ aaa\textbf{Bb}bccc \to (4)\ aaabbbccc$
    \end{itemize}

    \noindent (zárójelben az adott lépésre  alkallmazott szabály azonosítója, a helyettesíthető rész pedig kiemelt terminális, nem terminális).

    \noindent A $G$ grammatika által leírt nyelv: $L(G) := \Big\{a^nb^nc^n\ \Big|\ n > 0\Big\}$.

    \subsection*{Levezetés}

    \noindent \textbf{Mondatforma}: Mondatformának nevezzük terminális és nemterminális szimbólumok véges sorozatát: $(N \cup T)^{*}$.\\

	\paragraph*{Közvetlen levezetés}

	$ G = \Big<N,T,\mathcal{P},S\Big>$, és $u,v \in (N\cup T)^*$. A $v$ mondatforma közvetlenül (egy lépésben) levezethető $u$ mondatformaból $G$-ben, ha
	\[
        \textbf{u} = w_1\textbf{x}w_2, \ v = w_1\textbf{y}w_2\textrm{, ahol} \ \boldsymbol{x \rightarrow y} \in \mathcal{P} \quad (w_1, w_2 \in (N \cup T)^*)
	\]
	Jelölése: $ u \xrightarrow[G]{} v$\\

    \paragraph*{Közvetett levezetés}
							
    \noindent Azt mondjuk, hogy a $v$ mondatforma közvetetten levezethető az $u$ mondatformaból $G$-ben, ha létezik olyan $(\mathbb{N} \ni k \geq 0)$ szám és $x_0, x_1, \ldots, x_k \in (N \cup T)^{*}$ mondhatformák úgy, hogy
	\[
        \begin{array}{c}
          u = x_0 \\
          v = x_{k}
        \end{array}
         \quad \textrm{és} \quad x_i \xrightarrow[G]{} x_{i+1} \quad (1 \leq i \leq k-1)
	\]
	 						
    \noindent Másképp: A $v$ mondatforma levezethető az $u$ mondatformából G-ben (jele: $ u \xrightarrow[G]{}^{*} v $), ha $u=v$ vagy $\exists z \in  (N\cup T)^*$ mondatforma, hogy $ u \xrightarrow[G]{}^* z $ és $ z \xrightarrow[G]{} v $

	\paragraph*{Generált nyelv}

    \noindent A $G$ formális nyelvtan által generált nyelv szavai a kezdőszimbólumból levezethető szavak.

	\noindent $L(G)$ a $G=\Big<N,T,\mathcal{P},S\Big>$ grammatika által generált nyelv, ha:
	\[
		L(G) = \Big\{w \in T^{*}\ |\ S \xrightarrow[G]{}^{*} w\Big\}
	\]

    {\footnotesize \noindent {\color{blue} \faLightbulbO\ $\triangleright$ } }
    {\footnotesize
    \noindent \textbf{Ekvivalens nyelvtanok}: A $G_1$ és $G_2$ nyelvtanok ekvivalensek, ha $L(G_1) = L(G_2)$, azaz ugyanazt a nyelvet generálják. Jelölése: $G_1 \sim G_2$.\\

    \noindent \textbf{Kvázi ekvivalens nyelvtanok}: A $G_1$ és $G_2$ nyelvtanok kvázi ekvivalensek, ha $L(G_1)\backslash\{\varepsilon\}= L(G_2)\backslash\{\varepsilon\}$, azaz legfeljebb az üres szó tartalmazásában különböznek. Jelölése: $G_1 \begin{array}{c} \sim \\ kv \end{array} G_2$
    $\triangleleft$ \faLightbulbO}

	\subsection*{Generatív nyelvtanok osztályozása}
	
    \noindent A generatív nyelvtanok osztályozását a szabályok alakja alapján tehetjük meg.\\

	\noindent $G = \Big<N,T,\mathcal{P},S\Big>$ generatív grammatika $i$-típusú ($i=0,1,2,3$), ha $\mathcal{P}$ szabályhalmazára a következők teljesülnek:

	\begin{itemize}
		\item \emph{\textbf{Mondatszerű grammatika}} (i=0)\\ \\
		Tetszőleges, azaz $p\textbf{A}q \rightarrow \textbf{B}$ alakú, ahol, $A \in N,\ p,\ q,\ B \in (N \cup T)^{*}$
		\item \emph{\textbf{Környezetfüggő grammatika}} (i=1)\\ \\
		$p\textbf{A}q \rightarrow p\textbf{B}q$ alakú, ahol, $A \in N,\ p,\ q,\ B \in (N \cup T)^{*}$\\
        és $B \neq \varepsilon$, kivéve az $S \rightarrow \varepsilon$ szabály (ha létezik), ekkor $S$ nem fordul elő egyetlen szabály jobboldalán sem.
        
        Itt az A nemterminális szimbólum adott $p,\ q$ szavak esetén helyettesíthető egy $pAq$ mondatformában egy $B \in (N \cup T)^{+}$ szóval. Azaz, az $p = q = \varepsilon$ kivételével A helyettesítése B-vel, függ A környezetétől.
        
        Továbbá minden $\omega_1 \to \omega_2 \in \mathcal{P}$ szabályra (kivéve $S \rightarrow \varepsilon$): $l(\omega_1) \leq l(\omega_2)$
		\item \emph{\textbf{Környezetfüggetlen grammatika}} (i=2)\\ \\
		$A \rightarrow q$, ahol $A \in N,\ q \in (N \cup T)^{*}$.\\
		\item \emph{\textbf{Reguláris grammatika}} (i=3)\\ \\
		$A \rightarrow vB$ vagy $A \rightarrow v$ alakú, ahol, $A,B\in N,\ v \in T^*$.
    \end{itemize}

    \noindent A 3. típusú grammatikákat ill. nyelveket \emph{jobb lineáris}aknak is nevezzük, mivel minden szabály jobb oldalán legfeljebb egy nemterminális állhat és az is csak a jobb oldal végén.\\

    \noindent A különböző típusú nyelvtanok összességét \textbf{\emph{nyelvtani osztályoknak}} hívjuk és $\mathcal{G}_{i}$ jelöljük.
    \[
        \mathcal{G}_{i} := \big\{\text{i. típusú nyelvtanok}\big\}
    \]

    \noindent A definíciók alapján: $\mathcal{G}_3 \subseteq \mathcal{G}_2 \subseteq \mathcal{G}_1 \subseteq \mathcal{G}_0$.\\

    \noindent A nyelvtani osztályozást felhasználva a nyelveket is osztályozhatjuk:
    \[
        \mathcal{L}_{i} := \Big\{L\ \big|\ \text{L nyelv, és van olyan}\ G \in \mathcal{G}_{i},\ \text{amelyre}\ L(G) = L \Big\}
    \]

    \noindent Az előző nyelv hierarchia alapján $ \mathcal{L}_3 \subseteq \mathcal{L}_2 \subseteq \mathcal{L}_1 \subseteq \mathcal{L}_0 $ az ún. \textbf{\emph{Chomsky-féle hierarchia}}.\\

    \noindent Igaz a hierarchia erősebb változata is: $\mathcal{L}_3 \subset \mathcal{L}_2 \subset \mathcal{L}_1 \subset \mathcal{L}_0$.\\

    \noindent \textbf{Tétel}. Nem minden nyelv írható le nyelvtannal.\\

    {\footnotesize \noindent {\color{blue} \faLightbulbO\ $\triangleright$ } }
    {\footnotesize
    \noindent Miért érdekes a Chomsky-féle osztályozás?\\

     \noindent Az algoritmusok - mint ismeretes - problématípusok (problémaosztályok) megoldására szolgáló lépéssorozatok. A generatív grammatikákkal kapcsolatban több, számítógéppel (esetleg) megoldható probléma merül fel, melyekre algoritmusok készíthetők. Nagyon fontos annak eldöntése, hogy ha egy konkrét generatív grammatika egy konkrét problémájának  megoldására kifejlesztett algoritmus alkalmazható-e más grammatika ugyanazon problémájára, illetve mennyi módosítással alakítható át.\\

     \noindent Amennyiben a két grammatika ugyanazon Chomsky-osztályba tartozik, úgy egy jól megfogalmazott algoritmus váza a paraméterektől eltekintve elvileg alkalmazható lesz.\\

     \noindent Másik  fontos  indok,  ami  miatt  fontos  ismerni  egy  grammatika  Chomsky-osztályát,  hogy  igazolható  bizonyos  problémakörökről,  hogy  nem  készíthető  hozzá  általános  érvényű megoldó algoritmus.
    $\triangleleft$ \faLightbulbO}\\

    \noindent Az $\mathcal{L}_i \ (i=0,1,2,3)$ \emph{nyelvosztályok mindegyike zárt a nyelveken végezhető műveletekre nézve}. Tehát valahányszor $\mathcal{L}$-beli nyelveken végezzük el a műveletet, mindannyiszor $\mathcal{L}$-beli nyelvet kapunk eredményül. \\

    \noindent \textbf{Tétel}. (\emph{Kis Bar-Hillel lemma}): Tetszőleges $L \in \mathcal{L}_3$ nyelvhez $\exists n(L) > 0 \in \mathbb{Z}$, hogy $\forall u \in L,\ l(u) \geq n: u = xyz$ a következő tulajdonságokkal bír:
    \begin{itemize}
        \item $y \neq \varepsilon$
        \item $l(xy) \geq n$
        \item $\forall i = 0,1,\ldots: xy^{i}z \in L$
    \end{itemize}
    
    \noindent Azaz L nyelv minden szavának elég hosszú részszavában létezik elég rövid, nem üres, beiteralható részszó. pl. K-B HL $\Rightarrow HE \not \in \mathcal{L}_3$. HE: A $\left\{(,)\right\}$ ábécé feletti helyes zárójelezések halmaza.\\

    \noindent \textbf{Tétel}. (\emph{Zártság}): $\mathcal{L}_3$ zárt az unió, konkatenáció, lezárás, komplementerképzés, metszet, különbség és szimmetrikus differencia műveletekre nézve.\\

    {\footnotesize \noindent {\color{blue} \faLightbulbO\ $\triangleright$ } }
    {\footnotesize
    \noindent \textbf{Zsákutca}: Olyan nyelvtani jel, melyből az adott 2. típusú nyelvtanban nem vezethető le terminális szó.\\

    \noindent \textbf{Nem elérhető nyelvtani jel}: Olyan nyelvtani jel, mely az adott 2. típusú nyelvtanban semmilyen, a kezdőszimbólumból történő levezetésben nem szerepel.\\

    \noindent \textbf{Láncszabály}: Egy $G$ grammatika egy $x \to y \in \mathcal{P}$ szabálya láncszabály, ha $x,\ y \in N$, azaz nemterminálisok.\\

    \noindent \textbf{Láncszabálymentes nyelvtan}: Egy nyelvtan láncszabálymentes, ha nincs láncszabálya.\\

    \noindent \textbf{Epszilon-szabály}: Egy $G$ grammatika egy $x \to y \in \mathcal{P}$ epszilon-szabály, ha $y = \varepsilon$.\\

    \noindent \textbf{Korlátozott epszilon-szabály (KeS)}: Egy $G = \Big<N,T,\mathcal{P},S\Big>$ nyelvtanra teljesül a korlátozott epszilonszabály, ha nincsenek epszilonszabályai az $S \to \varepsilon$ szabály esetleges kivétel, de ez esetben minden $x \to y \in \mathcal{P}$ szabályra $y \in (N \cup T \backslash \{S\})^{*}$ (azaz semelyik szabály jobboldala sem tartalmazza a kezdőszimbólumot).
    $\triangleleft$ \faLightbulbO}\\    

    \noindent \textbf{Epszilonmentes nyelvtan}: Egy nyelvtan epszilonmentes, ha teljesül rá az epszilon-szabály.
						
	\section*{Automaták}

    Formális nyelvek megadása nemcsak generatív, hanem felismerő eszközökkel is lehetséges, azaz olyan számítási eszközök segítségével, amelyek szavak feldolgozására és azonosítására alkalmasak.
	Ilyen eszköz például az automata, amely egy szó, mint input hatására kétféleképpen viselkedhet: vagy \emph{elfogadja}, vagy \emph{elutasítja}.
	
    \subsection*{Véges automaták}
	
	\noindent \textbf{Definíció}. A véges automata egy rendezett ötös,
		\[
            \mathcal{A}=\Big<Q,T,\delta,q_0, F\Big>
        \]
		\begin{itemize}
			\item Q - állapotok véges, nemüres halmaza
			\item T - bementi szimbólumok ábécéje (véges)
			\item $\delta : Q \times T \rightarrow Q$ - állapot-átmeneti függvény
			\item $q_0 \in Q$ - kezdőállapot
			\item $F \subseteq Q$ - végállapotok halmaza
		\end{itemize}
							
	\noindent \emph{Működése}:\\

    \noindent A véges automata diszkrét időintervallumokban végrehajtott lépések sorozata által működik. Minden egyes lépés során az automata elolvassa a következő input szimbólumot és átmegy egy olyan állapotba, amelyet az állapotátmeneti függvény meghatároz (az aktuális állapot és input szimbólum alapján).\\
							
    \noindent Kezdetben az $A$ véges automata a $q_0$ kezdőállapotban van és az olvasófej az input szalagon levő $u \in T^*$ szó első betűjét dolgozza fel. Ezután a véges automata lépések sorozatát végrehajtva elolvassa az input $u$ szót; betűről betűre haladva olvas és új állapotba kerül.\\
							
    \noindent Miután az $u$ input szó utolsó betűjét is elolvasta a véges automata, vagy $q \in F,$ azaz elfogadó állapotba kerül, és akkor az $u$ szót az automata elfogadja, vagy az új állapot nem lesz eleme $F$-nek, és ekkor az automata a szót nem fogadja el.\\

    \noindent \textbf{Automata által felismert nyelv}: Az automata által felismert nyelv azon szavak halmaza, amelyekre a kezdő állapotból valamelyik végállapotba jut, azaz
    \[
        L(\mathcal{A}) = \Big\{u \in T^{*}\ \Big|\ \delta(q_0,u) \in F\Big\}.
    \]

    \paragraph*{Az automata megadási módszerei}

    \begin{itemize}
        \item \emph{Átmeneti gráf}: Olyan irányított gráf, melynek pontja az állapotok, és $A$ állapotból $t$-vel címkézett él fut $B$ állapotba, ha $\delta(A, t) = B$.
        \item \emph{Táblázat}: egyik tengelyén az állapotok, a másikon a terminálisok találhatók\\
        - lényegében az átmeneti gráf mátrixreprezentációja.
        \item \emph{Képlet}: a $\delta$ függvény megadása valamilyen zárt számítási képlettel, formulával.
    \end{itemize}

	\paragraph*{VDA - Véges determinisztikus automata}

    \noindent A $\delta$ függvény egyértékű, ezért minden egyes $(q, a)$ párra, ahol $(q, a) \in Q \times T$ egyetlen olyan $s$ állapot létezik, amelyre $\delta(q, a) = s$ teljesül. Ezért ezt a véges automatát determinisztikusnak nevezzük.\\

    \paragraph*{VNDA - Véges nemdeterminisztikus automata}

    \noindent Ha többértékű állapot-átmeneti függvényt is megengedünk, azaz $\delta : Q \times T \rightarrow 2^Q$, akkor nemdeterminisztikus véges automatáról beszélünk. (Ebben az esetben aktuális állapotnak egy állapothalmaz valamely elemét, mintsem egyetlen állapotot tekinthetünk.)\\
							
	\noindent Ez azt jelenti, hogy a kezdeti állapot helyettesíthető egy $Q_0 \subseteq Q$ kezdőállapot halmazzal.\\
	\noindent (És az is előfordulhat, hogy egy a input szimbólum esetén $\delta(q, a)$ üres az aktuális állapotok mindegyikére.)
	
    \paragraph*{Tulajdonságok}

    Az állapot-átmeneteket
	\[
        qa \rightarrow p
    \]
    alakú szabályok formájában is írhatjuk $p \in \delta(q, a) $ esetén. Jelöljük $M_\delta$-val az $A = \Big<Q, T, \delta,Q_0, F\Big>$ nemdeterminisztikus véges automata $\delta$ állapot-átmenet függvénye által az előbbi módon származó szabályok halmazát.\\
							
	\noindent Ha minden egyes $(q, a)$ párra egyetlen $qa \rightarrow p$ szabály van $M_\delta$-ban, akkor a véges automata determinisztikus, egyébként nemdeterminisztikus.
							
	\begin{itemize}
        \item Közvetlen redukció: \\
            Legyen $A = (Q, T, \delta, q_0, F)$ egy véges automata és legyenek $u, v \in QT^*$ szavak. Azt mondjuk, hogy az $A$ automata az $u$ szót a $v$ szóra redukálja egy lépésben/közvetlenül. Ha van olyan $qa \rightarrow p$ szabály $M_\delta$-ban, és van olyan $w \in T^*$ szó, amelyre $u = qaw$ és $v = pw$ teljesül.
			      						
        \item Redukció:\\
            Az $A = (Q, T, \delta, q_0, F)$ véges automata az $u \in QT^*$ szót a $v \in QT^*$ szóra redukálja $(u \Longrightarrow_A^* v)$, ha $u = v$, vagy $ \exists z \in QT^*$, amelyre $u \Longrightarrow_A^* z$ és $z \Longrightarrow_A v$ teljesül.
    \end{itemize}

    \noindent Az automata által elfogadott nyelv:\\
    \noindent Az $A = \Big<Q, T, \delta, q_0, F\Big>$ véges automata által elfogadott/felismert nyelv alatt az
    \[
        L(A) = \Big\{u \in T^* | \ q_0u \Longrightarrow_A^* p, \quad q_0 \in Q_0  \textrm{ és }  p \in F\Big\}
    \]
    szavak halmazát értjük.  (Az üres szó, akkor és csak akkor van benne az automata által elfogadott $L(A)$ nyelvben, ha $Q_0 \cap F \neq \emptyset$).\\
			      						
    \noindent \textbf{Tétel}. Minden $A$ nemdeterminisztikus véges automatához meg tudunk adni egy 3-típusú $G$ grammatikát úgy, hogy $L(G) = L(A)$ teljesül.\\
%        \ \\
%        \textbf{3. típusú nyelvtan konstrukciója automatából}: A konstrukció alapötlete a következő:
%        \begin{enumerate}
%            \item hozzuk normálformára a nyelvtant,
%            \item a nyelvtani jeleket feleltessük meg az automata állapotainak,
%            \item $q_0$-t rendeljük az S startszimbólumhoz,
%            \item $q \to tq' \in \mathcal{P} \Leftrightarrow \delta(q, t) = q'$,
%            \item $q \to \varepsilon \in \mathcal{P} \Leftrightarrow q \in F.$
%        \end{enumerate}
			      						
    \noindent \textbf{Tétel}. Minden 3-típusú G grammatikához meg tudunk adni egy $A$ véges automatát úgy, hogy $L(A) = L(G)$ teljesül.\\
			      						
    \noindent Ezek után fenáll a kérdés: Létezik-e olyan reguláris nyelv, amely VNDA-val felismerhető, de nem ismerhető fel VDA-val? \\
    \noindent Válasz: Nincs.
			      						
    \noindent \textbf{Tétel}. Minden $A = (Q, T, \delta,Q_0, F)$ VNDA-hoz  meg tudunk konstruálni egy $A' = (Q', T, \delta', q_0', F')$ VDA-t úgy, hogy $L(A) = L(A')$ teljesül.\\

    \noindent \textbf{Automata konstrukciója 3. típusú nyelvből.} A fentihez hasonló, fordított konstrukcióval készíthető 3. típusú nyelvtan alapján automata, ez azonban nem mindig lesz determinisztikus - ha nemdeterminisztikus állapotátmeneti függvényt kapunk, az automatát még determinisztikussá kell tennünk.\\

	\subsection*{Veremautomaták}

	\paragraph*{Definíció}

    Az egy veremmel rendelkező (1-verem) automatát a következő hetessel azonosítjuk
    \[
        A = \Big<Q, T, \Sigma, \delta, q_0, \sigma_0, F\Big>
    \]
    ahol
    \begin{itemize}
        \item Q - az állapotok véges halmaza
        \item T - input ábécé
        \item $\Sigma$ - veremábécé
        \item $\delta : Q \times (T \times \cup \{\varepsilon\}) \times \Sigma) \rightarrow 2^{Q \times \Sigma^{*}},\ |\delta(q,t,\sigma)| < \infty$ - átmeneti függvény
        \item $q_0 \in Q $ - kezdőállapot
        \item $\sigma_{0} \in \Sigma $ - a verem kezdőszimbóluma
        \item $F \subseteq Q$ - az elfogadó állapotok halmaza
    \end{itemize}

    {\footnotesize \noindent {\color{blue} \faLightbulbO\ $\triangleright$ } }
    {\footnotesize
    \noindent A veremautomata egy ütemben kiolvassa a központi egység állapotát, az input szó aktuális szimbólumát és a verem tetőelemét, ennek függvényében új állapotba kerül, a verem tetőelemét felülírja egy vagy több jellel (azaz egy szóval), az input szó következő betűjére áll az olvasófej (kivéve $\varepsilon$-mozgás) és a tetőmutató az új tetőelemre áll.\\

    \noindent A veremautomata konfigurációja alatt egy $uq$ alakú szót értünk, ahol $u \in Z^*$ a verem aktuális tartalma és $q \in Q$ az aktuális állapot. \\
    \noindent A kezdeti konfiguráció $\sigma_0q_0$.\\
							
    \noindent Működés: \\

    \noindent Tegyük fel, hogy az $A$ veremautomata olvasófeje az $a$ inputszimbólumon áll, a veremautomata $q$ állapotban van, valamint a verem tetején levő szimbólum $z$. Legyen $ \delta(z, q, a) = {(u_1, r_1), . . . , (u_n, r_n)}$, ahol $u_i \in Z^*$ és $r_i \in Q, 1 \leq i \leq n$. Ekkor $A$ következő állapota valamely $r_i$ lesz és egyidejűleg $z$-t helyettesíti az $u_i$ szóval, továbbá az olvasófej egy cellával jobbra lép az input szalagon.\\
    Ha $\delta(z, q, \varepsilon)$ nem üres, akkor ún. $\varepsilon$-átmenet hajtható végre.
							
    \noindent Ha az input szalag a $w \in T^*$ szót tartalmazza és a $z_0q_0$ kezdeti konfigurációból kiindulva a lépések sorozatát végrehajtva az $A$ veremautomata egy $up$ konfigurációba ér, ahol $p$ elfogadó állapot, akkor azt mondjuk, hogy $A$ elfogadta a $w$ szó.
    $\triangleleft$ \faLightbulbO}\\

    \noindent A veremautomata elfogadja a szót, ha üres a verem, és elfogadó állapotban van.\\
							
	\paragraph*{Tulajdonságok}

    \begin{itemize}
        \item Közvetlen redukció: \\
            $\alpha, \beta \in Z^*QT^*$\\
            Az $A$ veremautomata az $\alpha $ szót a $\beta$ szóra redukálja egy lépésben ($\alpha \Longrightarrow_A \beta$), ha:
            \[
                \exists z \in Z, \ p,q \in Q, \ a \in T \cup \{\varepsilon\}, \ r,u \in Z^* \textrm{ és } w\in T^*
            \]
            hogy:
            \[
                (u,p) \in \delta(z,q,a) \textrm{ és } \alpha = rzqaw \textrm{ és } \beta = rupw
            \]
		\item Redukció: \\
            Az A veremautomata az $\alpha$ szót a $\beta$ szóra redukálja ($\alpha \Longrightarrow_A^* \beta$), ha vagy $\alpha = \beta$, vagy
            $ \exists \gamma_1,...,\gamma_n \in Z^*QT^*$ szavakból álló véges sorozat, hogy $\alpha = \gamma_1, \ \beta = \gamma_n, \quad \textrm{ és } \quad \\ {\gamma_i \Longrightarrow_A \gamma_{i+1} \quad (i=1,...,n-1)}$
        \item A veremautomata által elfogadott nyelv: \\
            Az A veremautomata által (elfogadó állapottal) elfogadott nyelv:
            \[
                L(A) = \Big\{ w \in T^* | \ z_0q_0w \Longrightarrow_A^* up, \textrm{ ahol } u \in Z^*, p \in F \Big\}
            \]
		\item Determinizmus: \\
            A $\delta$ leképezést szabályok formájában is megadhatjuk. Az így nyert szabályhalmazt $M_\delta$-val jelöljük. Tehát
            \begin{enumerate}
                \item $zqa \rightarrow up \in M_\delta$ ha $(u, p) \in \delta(z, q, a)$
                \item $zq \rightarrow up \in M_\delta$ ha $(u, p) \in \delta(z, q, \varepsilon)$
            \end{enumerate}
			      								
            Az $A = (Z,Q, T,\delta, z_0, q_0, F)$ veremautomatát determinisztikusnak mondjuk, ha minden $(z, q) \in Z \times Q$ pár esetén:
            \begin{enumerate}								
                \item $\forall a \in T:  |\delta(z, q, a)| = 1$ és $\delta(z, q, \varepsilon) = \emptyset$
			      	      									
                vagy
                \item $|(z, q, \varepsilon)| = 1 $ és $\forall a \in T: \delta(z, q, a) = \emptyset$
            \end{enumerate}
        \item Üres veremmel elfogadott nyelv: \\
                Az $N(A)$ nyelvet az $A$ veremautomata üres veremmel fogadja el, ha
                \[
                    N(A) = \Big\{w \in T^* | \ z_0q_0w \Longrightarrow_A^* p, \textrm{ ahol } p \in Q \Big\}
                \]
		\end{itemize}

    \noindent \textbf{Tétel}. Tétel. Az 1-verem automatákkal felismerhető nyelvek osztálya megegyezik a 2. típusú grammatikák által generált nyelvek osztályával, azaz $\mathcal{L}_{V} = \mathcal{L}_{2}$.

	\section*{Reguláris nyelvek tulajdonságai és alkalmazásai}

    \noindent \textbf{Reguláris nyelv}: Reguláris nyelveknek nevezzük az alábbi három tulajdonsággal definiált ($\mathcal{L}_{REG}$) nyelvosztály elemeit:
    \begin{itemize}
        \item[(1)] $\mathcal{L}_{REG}$ tartalmazza az elemi nyelveket: $\emptyset, \{\varepsilon\},\{a\}\qquad (a \in U)$ (U: univerzális ábécé)
        \item[(2)] $\mathcal{L}_{REG}$ zárt az \emph{unió}, \emph{konkatenáció} és a \emph{lezárás} műveletekre.
        \item $\mathcal{L}_{REG}$ a legszűkebb olyan nyelvosztály, mely az $(1),\ (2)$ feltételeknek megfelel.\\
    \end{itemize}

    {\footnotesize \noindent {\color{blue} \faLightbulbO\ $\triangleright$ } }
    {\footnotesize\\
    \noindent \textbf{Reguláris kifejezés}: A reguláris kifejezések a reguláris nyelvek egyszerűsített leírását jelentik, ahol:
    \begin{itemize}
        \item az elemi reguláris kifejezések: $\emptyset, \varepsilon, a \qquad (a \in U)$
        \item ha $R_1$ és $R_2$ reguláris kifejezések, akkor $(R_1 \cup R_2), (R_1R_2),R_1^{*}$ is $V$ ábécé feletti reguláris kifejezések
        \item a reguláris kifejezések halmaza a legszűkebb, melyre az előbbi két pont teljesül\\
    \end{itemize}

    \noindent \textbf{V ábécé feletti reguláris kifejezések} (rekurzív definíció)
    \begin{itemize}
        \item az elemi reguláris kifejezések: $\emptyset, \varepsilon, a \qquad (a \in V)$
        \item ha $R_1$ és $R_2$ $V$ ábécé feletti reguláris kifejezések, akkor $(R_1 \cup R_2), (R_1R_2),R_1^{*}$ is reguláris kifejezések
        \item a $V$ ábécé feletti reguláris kifejezések halmaza a legszűkebb, melyre az előbbi két pont teljesül\\
    \end{itemize}

    \noindent \textbf{V ábécé feletti általánosított reguláris kifejezések} (rekurzív definíció)
    \begin{itemize}
        \item az elemi reguláris kifejezések: $\emptyset, \varepsilon, a \qquad (a \in V)$
        \item ha $R_1$ és $R_2$ $V$ ábécé feletti általánosított kifejezések, akkor $(R_1 \cup R_2), (R_1R_2), R_1^{*}, (R_1 \cap R_2), \overline{R_1}$ is $V$ ábécé feletti általánosított reguláris kifejezések
        \item a $V$ ábécé feletti általánosított reguláris kifejezések halmaza a legszűkebb, melyre az előbbi két pont teljesül\\
    \end{itemize}

    \noindent \textbf{Reguláris kifejezések szemantikája} (rekurzív definíció)
    \begin{itemize}
        \item az $\emptyset, \varepsilon, a$ reguláris kifejezések rendre az $\emptyset, \{\varepsilon\}, \{a\}$ nyelveket reprezentálják
        \item ha $R_1$ az $L_1$ és $R_2$ az $L_2$ nyelvet reprezentálja, akkor $(R_1 \cup R_2),\ (R_1R_2),\ R_1^{*}$ rendre az $L_1 \cup L_2,\ L_1L_2,\ L_1^{*}$ nyelveket reprezentálja
    \end{itemize}

    \noindent \textbf{Reguláris kifejezések általánosított szemantikája} (rekurzív definíció)
    \begin{itemize}
        \item az $\emptyset, \varepsilon, a$ reguláris kifejezések rendre az $\emptyset, \{\varepsilon\}, \{a\}$ nyelveket reprezentálják
        \item ha $R_1$ az $L_1$ és $R_2$ az $L_2$ nyelvet reprezentálja, akkor $(R_1 \cup R_2),\ (R_1R_2),\ R_1^{*}, \overline{R_1}$ rendre az $L_1 \cup L_2,\ L_1L_2,\ L_1^{*}, \overline{L_1}$ nyelveket reprezentálja
    \end{itemize}

	\paragraph*{Axiómák}

    \noindent $P,Q,R$ reguláris kifejezések. Ekkor fennállnak a következő tulajdonságok:
    \begin{itemize}
        \item Asszociativitás:
            \[
                P+(Q+R) = (P+Q)+R
            \]
            \[
                P\cdot(Q\cdot R) = (P\cdot Q)\cdot R
            \]
			\item Kommutativitás:
            \[
                P+Q = Q+P
            \]
			\item Disztributivitás:
            \[
                P\cdot (Q+R) = P\cdot Q + P\cdot R
            \]
			\[
                (P+Q)\cdot R = P\cdot R + Q\cdot R
            \]
			\item Egységelem:
            \[
                \varepsilon\cdot P = P\cdot\varepsilon = P
            \]
            \[
                P^* = \varepsilon + P \cdot P^*
            \]
			\[
                P^* = (\varepsilon+P)^*
            \]
        \end{itemize}
							
    \noindent A fenti axiómák azonban még önmagukban nem elegendőek az összes reguláris kifejezés előállítására (helyettesítés segítségével). Szükség van még az alábbi inferencia szabályra:
    \[
        P=R+P\cdot Q \quad \land \quad \varepsilon \notin Q \quad \Longrightarrow \quad P = R\cdot Q^*
    \]
							
    \noindent Vegyük még hozzá az $\emptyset$ szimbólumot a reguláris kifejezések halmazához, amely az üres nyelvet jelöli. (Ebben az esetben nincs szükségünk az $\varepsilon$ szimbólumra, mivel $\emptyset^* = \{\varepsilon\}$). Így, a definícióban helyettesíthetjük az $\varepsilon$ szimbólumot az $\emptyset$ szimbólummal. Ekkor helyettesítjük $\varepsilon$-t a megelőző axióma rendszerben $(\emptyset)^*$-gal és még egy további axiómát tekintünk:
    \[
        \emptyset\cdot P = P\cdot\emptyset = \emptyset
    \]

    \noindent A fenti szabályok elégségesek ahhoz, hogy levezessünk minden érvényes egyenlőséget reguláris kifejezések között.
	
    \paragraph*{Reguláris kifejezések és reguláris nyelvek}

    \noindent Minden reguláris kifejezés egy reguláris (3-típusú) nyelvet jelöl, és megfordítva, minden reguláris nyelvhez megadható egy, ezen nyelvet jelölő reguláris kifejezés.\\
    $\triangleleft$ \faLightbulbO}\\

    \noindent \textbf{Tétel}. (\emph{Kleene-tétel}): $\mathcal{L}_{REG} = \mathcal{L}_{3}$, azaz a reguláris nyelvek osztálya megegyezik a véges determinisztikus automatával előállítható nyelvek osztályával.\\

    \noindent \textbf{Tétel}. Az $\mathcal{L}_3$ nyelvosztály zárt a komplementer, a metszet, a különbség és aszimmetrikus differencia műveletekre, a zártsági tétel miatt pedig zárt az unió, konkatenáció és a lezárás műveletekre is.\\

    \subsection*{Reguláris nyelvek felhasználási területei\\}

    Bár a reguláris nyelvtanok az összes nyelvek viszonylag szűk osztályát jelölik ki, könnyű kezelhetőségük miatt gyakorlati alkalmazásuk rendszeres.\\

    \noindent \textbf{\emph{Lexikális elemző}} (scanner): A fordítóprogramok legalsó szintjén, a lexikális elemek felderítésében nagy szerepet játszanak a 3. típusú nyelvtanok: általában reguláris kifejezésekkel azonosítják a nyelv lexikális elemeit.\\

    \noindent \textbf{\emph{Mintaillesztés}}: Mintaillesztési feladatokban (pl. adatok között/szövegben történő keresés) reguláris kifejezéssel írható le az illesztendő minta. Ebből általában véges determinisztikus automatát (például Knuth-Morris-Pratt automatát) generálnak.\\
	
    {\footnotesize \noindent {\color{blue} \faLightbulbO\ $\triangleright$ } }
    {\footnotesize
	\subsection*{Lineáris grammatikák és nyelvek}
	
    \paragraph*{Definíció}

    \noindent Egy $\Big<N,T,\mathcal{P},S\Big>$ környezetfüggetlen grammatikát lineárisnak nevezünk, ha minden szabálya:
    \begin{enumerate}
        \item $A\rightarrow u, \quad A \in N, u \in T^*$
        \item $A\rightarrow u_1Bu_2, \quad A,B\in N, u_1,u_2 \in T^*$
    \end{enumerate}

    \noindent Továbbá $G$-t bal-lineárisnak, illetve jobb-lineárisnak mondjuk, ha $u_1 = \varepsilon$, illetve $u_2 = \varepsilon$ minden 2. alakú szabályra.
								
    \noindent Egy $L$ nyelvet lineárisnak, bal-lineárisnak, illetve jobb-lineárisnak mondunk, ha van olyan $G$ lineáris, bal-lineáris, illetve jobb-lineáris grammatika, amelyre $L = L(G)$ teljesül.

    \paragraph*{Lineáris és reguláris grammatikák, nyelvek}

    \noindent A jobb-lineáris grammatikák azonosak a reguláris grammatikákkal (3-típusúak).
								
    \noindent \textbf{Tétel}. Minden bal-lineáris grammatika reguláris (3-típusú) nyelvet generál.\\
    $\triangleleft$ \faLightbulbO}\\
	
	\section*{Környezetfüggetlen nyelvek tulajdonságai és elemzésük}

	\subsection*{Környezetfüggetlen grammatikák normálformái}

	\noindent Környezetfüggetlen grammatikák normálformái olyan grammatikai transzformációval előállított  grammatikák, melyek:
	\begin{itemize}
		\item bizonyos szintaktikai feltételeknek/tulajdonságoknak tesznek eleget
		\item (általában) valamilyen szempontból egyszerűbbek, mint az eredeti grammatikák
		\item ugyanazt a nyelvet generálják (így ugyanazon típusba tartoznak)
	\end{itemize}
	
    \noindent \textbf{Tétel}. Minden $\Big<N,T,\mathcal{P},S\Big>$ környezetfüggetlen grammatikához meg tudunk konstruálni egy vele ekvivalens $G' = \Big<N',T,\mathcal{P}',S'\Big>$ környezetfüggetlen grammatikát úgy, hogy: $G'$ minden szabályának jobboldala nemüres szó [kivéve azt az esetet, mikor $\varepsilon \in L(G)$, ekkor $S' \rightarrow \varepsilon$ az egyetlen szabály, melynek jobboldala az üres szó és ekkor $S'$ nem fordul elő $G'$ egyetlen szabályának jobboldalán sem.]
	
    \paragraph*{$\varepsilon$-mentes grammatika}

    \noindent A $G$ grammatika $\varepsilon$-mentes, ha egyetlen szabályának jobboldala sem az üres szó.\\
							
    \noindent \textbf{Tétel}. Minden környezetfüggetlen $G$ grammatikához meg tudunk konstruálni egy $G'$ $\varepsilon$-mentes környezetfüggetlen grammatikát, amelyre $L(G') = L(G) - \{\varepsilon\}$ teljesül.\\

    \paragraph*{Chomsky normálforma}

	\noindent A $G = \Big<N,T,\mathcal{P},S\Big>$ környezetfüggetlen grammatikát Chomsky-normálformájúnak mondjuk, ha minden egyes szabálya:
	\begin{itemize}
        \item $X \rightarrow a$, ahol $X \in N, a \in T$
        \item $X \rightarrow YZ$, ahol $X,Y,Z \in N$
    \end{itemize}
	
    \noindent \textbf{Tétel}. Minden $\varepsilon$-mentes $\Big<N,T,\mathcal{P},S\Big>$ környezetfüggetlen grammatikához meg tudunk konstruálni egy vele ekvivalens $G' = \Big<N',T,\mathcal{P}',S'\Big>$ Chomsky-normálformájú környezetfüggetlen grammatikát.\\
							
    \noindent \textbf{Tétel}. (az előző következménye): Minden $G$ környezetfüggetlen grammatika esetében eldönthető, hogy egy $u$ szó benne van-e $G$ grammatika által generált nyelvben.\\

	\paragraph*{Redukált grammatika}

	\begin{itemize}
        \item A környezetfüggetlen grammatika egy nemterminálisát \textbf{\emph{inaktívnak/nem aktívnak}} nevezzük, ha nem vezethető le belőle terminális szó.
        \item A környezetfüggetlen grammatika egy nemterminálisát \textbf{\emph{nem elérhetőnek}} nevezzük, ha nem fordul elő egyetlen olyan szóban sem, amely a kezdőszimbólumból levezethető.
%        \item Egy nemterminálist nem hasznosnak mondunk, ha vagy inaktív, és/vagy nem elérhető.
        \item Az, hogy egy $A$ nemterminális elérhető-e vagy aktív-e, az eldönthető.
        \item \emph{Egy környezetfüggetlen grammatika redukált}, ha minden nemterminálisa \emph{aktív} és \emph{elérhető}.
    \end{itemize}
	
    \paragraph*{Elemzés}

    \noindent A környezetfüggetlen nyelvtanok segítségével szintaxisfát építhetünk. Sikeres szintaxisfafelépítés esetén az elemzés sikerült, különben szintaktikai hibát találunk. Módszerek: legbal levezetés, legjobb levezetés.\\

    \noindent \textbf{Tétel}. Minden környezetfüggetlen grammatikához meg tudunk konstruálni egy vele ekvivalens redukált környezetfüggetlen grammatikát.
	
	\subsection*{Levezetési fa}

    \noindent A környezetfüggetlen grammatikák levezetéseit fákkal is jellemezhetjük. A levezetési fa egy szó előállításának lehetőségeiről ad információkat. A levezetési fa egy irányított gráf, amely speciális tulajdonságoknak tesz eleget:
	\begin{itemize}
		\item A gyökér cimkéje: $S$
		\item A többi csúcs cimkéje ($N\cup T$) valamely eleme
	\end{itemize}

    \noindent A levezetési fa nem minden esetben adja meg a levezetés során alkalmazott szabályok sorrendjét. Két levezetés lényegében azonos, ha csak a szabályok alkalmazásának sorrendjében különbözik.\\
					
    \noindent Egy környezetfüggetlen grammatika minden levezetési fája egy egyértelmű (egyetlen) legbaloldalibb levezetést határoz meg. A legbaloldalibb levezetés során minden levezetési lépésben a legbaloldalibb nemterminálist kell helyettesítenünk.\\

	\noindent \textbf{Tétel}. Minden környezetfüggetlen grammatikáról eldönthető, hogy az általa generált nyelv az üres nyelv-e vagy sem.

	\subsection*{Bar-Hillel Lemma}

    \noindent Minden $L \in \mathcal{L}_2$ (környezetfüggetlen nyelvhez) meg tudunk adni két nyelvfüggő $p$ és $q$ egész konstanst ($p=p(L),\ q = q(L)$), amelyekre ha $u \in L$ és $|u| > p$, akkor $u-nak$ létezik a következő felbontása
    \[
        u = xyzvw \qquad (u,x,w,y,v \in T^*),\ \text{melyre}
    \]
	\begin{itemize}
        \item $|yzv| \geq q$,
        \item $|yv| > 0$,
        \item $xy^{i}zv^{i}w \in L \qquad (\forall i \geq 0)$
    \end{itemize}

    \noindent A Bar-Hillel lemma azt mondja ki, hogy egy végtelen környezetfüggetlen nyelvben minden elég hosszú szóhoz végtelen sok "hasonló szerkezetű szó" van.\\

    \noindent $G$ generált $L$ nyelv esetén, ha $|N| = l$, akkor $L$-beli szavak hossza legfeljebb $2^l$.\\
	
    \section*{Kiegészítés}

    \begin{tabular}{l|c}
        Minden környezet független nyelv felismerhető veremautomatával & $\checkmark$ \\ \hline \\
        Minden környezet független nyelv felismerhető 1 veremmel & $\checkmark$ \\ \hline \\
        Minden 3. típusú nyelv felismerhető VNDA-val & $\checkmark$ \\ \hline \\
        Minden 0-veremautomata véges determinisztikus automata & $\checkmark$ \\ \hline \\
        Minden rekurzív nyelv rekurzívan felsorolható & $\checkmark$ \\ \hline \\
        Minden (kiterjesztett)3. típusú nyelvtan 1. típusú nyelvtan & \checkmark \\ \hline \\
        Minden nyelvtannal leírt nyelv parciálisan rekurzív & \checkmark \\ \hline \\
        Minden rekurzívan felsorolható nyelv parciálisan rekurzív & \checkmark \\ \hline \\
        Minden Chomsky 1. típusú nyelv rekurzív nyelv & \checkmark \\ \hline \\
        Minden Chomsky 2. típusú nyelv rekurzívan felsorolható nyelv & \checkmark \\ \hline \\
        Minden Chomsky 2. típusú nyelv rekurzív nyelv & \checkmark \\ \hline \\
        Minden parciálisan rekurzív nyelv környezet\-függő & $\bigotimes$ \\ \hline \\
        Minden parciálisan rekurzív nyelv reguláris & $\bigotimes$ \\ \hline \\
        Minden nyelvtannal leírt nyelv rekurzívan felsorolható & $\bigotimes$ \\ \hline \\
        Minden nyelvtannal leírt nyelv reguláris & $\bigotimes$ \\ \hline \\
        Minden nyelvtannal leírt nyelv rekurzív & $\bigotimes$ \\ \hline \\
        Minden nyelvtannal leírt nyelv felismertethető VDA-val & $\bigotimes$ \\ \hline \\
        Minden Chomsky 1. típusú nyelv felismerhető 1 veremmel & $\bigotimes$ \\ \hline \\
        Minden Chomsky 1. típusú nyelv felismerhető VDA-val & $\bigotimes$ \\ \hline \\
        Minden rekurzívan felsorolható nyelv reguláris & $\bigotimes$ \\ \hline
    \end{tabular}

\end{document} 